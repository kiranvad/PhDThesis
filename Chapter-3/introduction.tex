Cyclic Voltammetry~(CV) is an electro-chemical characterization technique that measures current generated under a cyclic voltage load between a initial and final voltages varied at a given rate.
The measured current is a highly non-linear response from various physical phenomenon such as mass transport, kinetics, adsorption etc. 
In principle, it is possible to determine the properties associated with the underlying physical phenomenon. 
However, the property extraction is a non-trivial task. 
In a CV experiment, a steady state current is obtained when all reactions in the mechanism have the same apparent rate constants~\cite{BardFaulkner}. 
This is because the facile reactions in the sequence are held back from their maximum rates by the sluggish reactions called a \textit{rate determining step} that also determines the magnitude of steady state current.   
Extracting rate constants of the rate determining step thus requires the CV curve to be in a S-shape~\cite{costentin2015cyclic, rountree2014evaluation} with a clear steady state current region resolved during measurement. 
Towards this goal, obtaining a S-shaped CV curve requires the experiment to be run with a set of conditions~(e.g, temperature, substrate concentration, scan rate), amenable for S-shape CV curves which are unknown \textit{a priori}.
Moreover, choosing conditions where a given electrochemical system exhibits a S-shaped CV curve is dependent on underlying system of electrochemical reaction(s) which is(are) also unknown for novel materials. 
In the absence of a known mechanism, an exhaustive search over all the possible tunable parameters is performed~\cite{martin2016qualitative} to narrow down the region of interest. 
Such exhaustive strategy comes at a price of a very high computational cost especially in high-dimensional search spaces of multiple complex reaction mechanisms.

As an alternative approach, experts define a figure of merit~(FOM)~\cite{rountree2014evaluation}(a performance measure) as a proxy signature of a physical phenomenon of interest. 
FOM extracted from CV can be also used in material discovery using data-driven methods. 
For example, in~\cite{stein2019functional,li2019application,haber2014high,suram2015generating} different types of FOM have been used for catalyst discovery using data-driven methods.
With FOM defined, the goal is to find a material that produces a response with a FOM that is better than that of known materials.
For instance, in case of a high-throughput exploration for a new catalyst, the overpotential is a common FOM~\cite{norskov2004origin}(or performance measure) used in the combinatorial searches~\cite{suram2015generating}. 
The over-potential can be thought of as the voltage~(beyond the thermodynamic requirement) required to produce a (pre-defined) target current.
This FOM has clear utility to screen for well performing materials, but misses on the main advantage of CV - that is the capability to extract the kinetic information~(such as rate constants~\cite{FOWA}, turn-over frequencies~\cite{TOF,FOWA2}).

Given the time and financial constraints, an active learning method~\cite{jiang2018efficient,RGActiveIntro, GardnerALIntro} is proposed to accelerate the process of extracting kinetic information from CV curves. 
Rather than relying on the selection of figure of merit, a function space representation of our target~(S-shaped) and non-target~(everything else) CV responses is used in the Bayesian Model Selection (BMS) for automatic classification. 
We encode prior knowledge of target and non-target CV responses using the basis functions of a Gaussian processes~(\(\mathcal{GP}\)) function space.
\(\mathcal{GP}\) have been previously used to infer the kinetic parameters~\cite{gavaghan2018use,robinson2018separating} of a CV response by using a maximum likelihood estimate and \(\mathcal{GP}\) regression. 
In another work ~\cite{li2019application}, a Bayesian approach is used to search for an approximate rate constant when the reaction mechanism is known. 
In this work, however, \(\mathcal{GP}\) is used as a data representation model to distinguish S-shaped CV curves from other types of continuous CV curves. 
Once a S-shaped CV curve is collected, the foot-of-the-wave analysis~(FOWA)~\cite{FOWA} can be used to extract the rate constant of a rate determining step. 
When combined together with FOWA, the proposed approach can be a robust technique that does not require any knowledge of the actual reaction mechanism.

In this chapter, we focus on S-shaped CV responses due to their utility for:~a) extracting kinetic information~\cite{costentin2015cyclic, rountree2014evaluation}--using the foot of the wave analysis~\cite{FOWA} that can only be applied to a S-shaped CV curve.~b) screening for bi-functional catalysts -- materials that produce CV curves similar to S-shape in two different voltage sweep ranges~\cite{bradley2019reversible, jung2016optimizing}. 
While the two applications are different, they can be approached under a common framework of active search to find S-shaped CV curves within the combinatorial space. 


The rest of the chapter is organized as follows: (i)~First  Bayesian active learning framework is introduced with a general probabilistic model. Then a connection is established between collected data at observed locations with the oracle used to classify and update the decision model used for active learning. (ii)~Bayesian Model Selection~(BMS) procedure is then introduced to compute a classification preference for targets and non-targets based on collected data and a set of parametric models. (iii)~a \(\mathcal{GP}\) model is introduced from a function space representation point of view to be used as a parametric model in BMS.  (iv)~The methodology is then applied to a search space of a simple EC mechanism to demonstrate the application of the BMS oracle to classify CV responses in order of its S-shape. (v)~Finally, the BMS oracle is used in active search to address the challenges in knowledge extraction, virtual screening of materials for electrochemical applications using cyclic voltammetry.