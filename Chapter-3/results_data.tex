\subsection{Data generation}\label{sec:data}
The EC mechanism is a two step reaction comprising of an electron transfer~\Cref{E} followed by a chemical reaction in~\Cref{C}. 
\begin{chequation}
\begin{align}
    &\ce{P + e <=> Q} \label{E}\\
    &\ce{Q + A -> P } \label{C}
\end{align}
\end{chequation} 
Electro-chemical kinetics of the EC mechanism can be modeled and solved using governing partial differential equations~\cite{saveant1984homogeneous}. 
In this section, we are interested in modeling the (micro)kinetics of species (and electron) that contributes to current generation under cyclic voltage sweep at a given sweeping rate.

Towards this goal, the transport of the three species (P, Q, A) in the solution is modelled using Fick's second law of diffusion with a source term corresponding to the heterogeneous reactions:
\begin{align}
    \pdv{C_P}{t} &= D_{\text{diff}}\pdv[2]{C_P}{u} + k_{s}C_{Q}C_{A} \nonumber\\
    \pdv{C_Q}{t} &= D_{\text{diff}}\pdv[2]{C_Q}{u} - k_{s}C_{Q}C_{A} \nonumber\\
    \pdv{C_A}{t} &= D_{\text{diff}}\pdv[2]{C_A}{u} - k_{s}C_{Q}C_{A} \label{eq:fickslaws}
\end{align}
with the boundary conditions defined as follows:
\begin{align}
    t&=0, \forall u &  &C_P = C^0_{P}, C_A = C^0_{A}, C_Q = C^0_{Q} \nonumber \\
    t&>0, u\rightarrow \infty &  &C_P = C^0_{P}, C_A = C^0_{A}, C_Q = C^0_{Q}  \nonumber\\
    t&>0, \forall u &  &\pdv{C_A}{u}=0; \pdv{C_P}{u}+\pdv{C_Q}{u}=0; C_P/C_Q = \exp(\frac{F}{RT}(V-E^0)) \label{eq:bcs}
\end{align}
In~\Cref{eq:fickslaws,eq:bcs} the formal reversible potential of electron transfer reaction~\cref{E} is \(E^0\), the  concentration of catalyst P is \(C_P\), specie Q is \(C_Q\) and substrate A is \(C_A\). \(D_{\text{diff}}\) is a common diffusion coefficient for all species and \(k_{s}\) is the rate constant of the forward reaction in~\Cref{C}. 
The spatial domain is denoted as \(u\) starting from the working electrode~(i.e. \(u=0\)) assuming a semi-infinite domain. The time scale of the simulation is denoted as \(t\). Initial concentrations~(i.e. at \(t=0\)) are denoted with a superscript 0. \(V\) represents the time varying applied voltage. 
For a cyclic voltage sweep between voltages \(V_i, V_f\) at a rate of \(\nu~V/s\) we get~\Cref{eq:voltage_scan} for V~(\(T_s\) is switching time).
\begin{align}\label{eq:voltage_scan}
    V(t) = 
    \begin{cases}
        V_i + \nu t & 0<t<T_s\\
        V_f - \nu t & T<t<2T_s
    \end{cases}
\end{align}
Digital simulation of system of partial differential equations in~\Cref{eq:fickslaws,eq:bcs} is performed to determine spatio-temporal concentration profiles of species P, Q, A. 
The Faradaic current observed during the cyclic voltage load is computed using~\Cref{eq:current} following the Butler-Volmer model for heterogeneous electron transfer at the electrode surface.
\begin{align}\label{eq:current}
    i(t,v) = FA_{\text{surf}}k^0\left[C_Q\exp(\frac{\alpha F}{RT}(V-E^0)) - C_P\exp(\frac{(1-\alpha)F}{RT}(V-E^0))\right]
\end{align}
In~\Cref{eq:current}, \(F\) is Faraday's constant, \(A_{\text{surf}}\) is surface area of electrode~( \(=1~cm^2\)), \(R\) is universal gas constant, \(T\) is room temperature.
\(k^0\) is heterogeneous electron transfer rate constant and \(\alpha\) is a symmetric charge transfer coefficient~(=0.5).
    
We use the freeware software MECSim~\cite{kennedy2015monash,MECSim} to digitally simulate the cyclic voltammetry response in the voltage range of \([-0.5V,0.5V]\). 
Along with the parameters used in~\Cref{eq:fickslaws,eq:bcs}, MECSim can also simulate the effects of an uncompensated resistance (\(R_u\)), double layer capacitance (\(C_{dl}\)) which are not used in this example case study. 
We form a 6-dimensional design search space using \(C_P^0, C^A_0, k_{s}, k^0, \nu, E^0\) and set the values of \(R_u = 0, C_{dl} = 0, \alpha=0.5, D=1\times10^{-5}\). 
\Cref{tab:search_space} lists the combinatorial space defined with six design variables (dimensions of search space) and number of samples along the dimension used to create an exhaustive search grid of tunable settings.
After excluding responses from a diverging simulation arising from a combination of non-physical parameters for MECSim~\footnote{see MECSim documentation for known limitations} we get a total of \(\approx 17\times10^3\) CV curves in our database. 
\begin{table}[h]
\centering
\begin{tabular}{|c|c|c|}
\hline
Parameter & range & number of levels per dimension \\  \hline
\(\log C_P^0 \) &[-2,3] & 5\\ \hline
\(\log C_A^0 \) &[-2,3] & 5\\ \hline
\( E^0 \) &[-0.4,0.4] & 5\\ \hline
\(\log k_{s} \) &[-1,6] & 5\\ \hline
\(\log k^0 \) &[-1,6] & 5\\ \hline
\(\log \nu \) &[-2,4] & 6\\ \hline
\end{tabular}
     \caption{Combinatorial space used to generate CV responses in EC mechanism along with number of levels used in the exhaustive search.}
     \label{tab:search_space}
 \end{table}
 

