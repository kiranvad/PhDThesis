\section{Bayesian Model Selection}
The BMS is a tool to identify a preferred model from a family of parametric probability distributions, each of which can explain the observed data with differing degrees of fidelity. 
Using a supervised learning procedure that compares an input \(\textbf{X}\) and output \(\textbf{y}\), we compute a model posterior using Bayes rule to select the model that best explains the observed data \(\mathcal{D}=(\textbf{X},\textbf{y})\). 

Here, given the observed data \(\mathcal{D}=(\textbf{X}, \textbf{y})\), we compute probability that the data is sampled from any given model encoding our prior information. 
Computed posterior probabilities will be used as a score to differentiate whether the collected response (for example, a CV response at any input location in the search space of materials) is a target~(with higher probability for the corresponding target model) or not. 
In this chapter, we use both BMS and active learning in a related but different context. 
BMS is used with observed data encoding a single CV curve while active learning is used in the search space with their corresponding binary labels~(i.e. a target or not) as observed data. 
Moreover, BMS is used as an oracle for the active learning task with models \(\mathcal{M}_j\) as \(\mathcal{GP}\).

For each model \(\mathcal{M}\) with a parameter index \(\theta\)-- a concatenated vector of hyper-parameters-- we first compute model evidence \(p(\textbf{y}\vert \textbf{X},\mathcal{M})\) on the observed data \(\mathcal{D}=(\textbf{X}, \textbf{y})\):
\begin{equation}
    p(\textbf{y}\vert \textbf{X},\mathcal{M}) = \int p(\textbf{y}\vert \textbf{X},\theta,\mathcal{M})p(\theta\vert \mathcal{M}) \diff \theta
\label{eq:evidM}
\end{equation}
where \(p(\textbf{y}\vert \textbf{X},\theta,\mathcal{M})\) is probability of obtaining outputs \(\textbf{y}\) given input data \(\textbf{X}\) and a model \(\mathcal{M}\). \(p(\theta\vert \mathcal{M})\) represents distribution of parameter \(\theta\) for any given model \(\mathcal{M}\).

To understand which model to prefer from a finite set of models  \(\bigl\{\mathcal{M}_i\bigr\}_{i=1}^n\), we apply the Bayes rule to compute the posterior probability of each model \(\mathcal{M}_{j}(j\in\{1,2,..n\})\) given data \(\mathcal{D}\) using the posterior of~\Cref{eq:evidM}: 
\begin{equation}
    p(\mathcal{M}_j\vert \mathcal{D}) = \frac{p(\textbf{y}\vert \textbf{X},\mathcal{M}_j)p(\mathcal{M})}{p(\textbf{y},\textbf{X})}
    \label{eq:postM}
\end{equation}
where \(p(\mathcal{M})\) represents a prior over the finite set of models that is typically taken to be uniform i.e. no prior preference to any single model. 
One common approach is to use logarithm of the probability which can be interpreted as the information content of a probability model given data. 
Taking the logarithm of~\Cref{eq:postM}, we get the following~(See Supplementary Information): 
\begin{equation}
    \log p(\mathcal{M}_j\vert \mathcal{D}) = -\log \left [ 1+\sum_{i\neq j}^{n} \frac{p(\textbf{y}\vert \textbf{X},\mathcal{M}_i)}{p(\textbf{y}\vert \textbf{X},\mathcal{M}_j)}\right]
\label{eq:logPostM}
\end{equation}




