\chapter{Conclusion}\label{chap:conclusion}

% \vspace{-5pt}
% \section{Conclusion}\label{sec:concld}
While there are many potential and emerging applications for multi-agent drone networks, deployment and testing of such systems is extremely challenging because it requires experience in networking, software systems, robotics, mission planning, and control among others. To mitigate these challenges, in this dissertation, we presented the University at Buffalo's Airborne Networking and Communications testbed (UB-ANC), which aims to facilitate the design of multi-drone networks and applications. UB-ANC comprises three components: the UB-ANC Drone, the UB-ANC Emulator, and the UB-ANC Planner. The UB-ANC Drone is a flexible open drone platform that provides tools for researchers to test and evaluate different mission planning algorithms and network protocols on actual drones. The UB-ANC Emulator aims to make it easy and convenient to design, implement, test, and debug distributed multi-agent mission planning algorithms in software to ensure correct system operation prior to experimentation in the field on actual UB-ANC drones. Finally, UB-ANC Planner is an energy-efficient coverage path planner, which aims to minimize the maximum energy consumption among drones covering an arbitrary area with obstacles. All projects are open source and available online at 
\begin{center}
{\tt \url{https://github.com/jmodares}}.
\end{center}

