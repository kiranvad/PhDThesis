\chapter{Conclusion}
The goal of the thesis has been to evaluate the use of data representations for application of machine learning to material discovery.
We introduced a distance measure learning approach into the data analytics for compositional libraries. 
A metric was learned from the data while leveraging the composition information about the design space. 
This is in contrast to the exhaustive search of distance measure and provides a automated way of selecting a metric.
We demonstrated the robustness of this approach for two case studies. 
First, we evaluated the framework using synthetically generated data sets of cyclic voltammetry measurements. 
Next, we applied the framework to labeled XRD data and constructed the phase diagram.
We showed that our method has the ability to identify regions in the design space where a systematic variation in design variable gives rise to systematic variation in response. 
Using a model electrochemistry data, we showed that a distance measure learned using our approach is at least as good as the metric chosen using an exhaustive search.
Through various test cases, we have shown that learning a similarity measure using MT-LMNN has the ability to automatically quantify the similarity measure for any given data set. This makes MT-LMNN a better alternative to an exhaustive search of complex and data specific metrics as well as a promising approach for automated distance learning for compositional libraries. 


In the second research task, we defined and evaluated a \(\mathcal{GP}\)-based oracle for materials discovery using cyclic voltammetry.
Next, we combined the oracle with a state-of-the-art active batch search to identify condition resulting in the targeted shape of CV curve. 
We demonstrated a robust high throughput combinatorial search to find the target responses using only \(<6\%\) of total number of CV experiments from the corresponding exhaustive search (with a discrete sampling of modest 5 levels per dimension). 
This work has implications in identification of characterization conditions where kinetic knowledge extraction from the cyclic voltammetry can be preformed more effectively. 
Specifically, we have illustrated a framework that can be used to identify S-shaped CV curves. 
Once S-shaped CV curve is obtained, a foot of the wave analysis can be applied ~\cite{FOWA} to extract rate constant for rate determining step, overpotential dependent turn over frequency etc.
In this sense our method has applications in accelerated knowledge extraction, with the application in screening for target catalysts including the bi-functional alkaline fuel cell catalysts that motivated this work.


A data exploratory analysis is introduced for phase diagram generation with the capabilities to evaluate for physical and computational accuracy. 
We use a data-analysis pipeline to perform initial screening of materials to identify subgroups there by decreasing the number of samples to be studied by an expert.
We apply our methodology on two sets of material systems and provide a framework to derive design rules for solvent selection in OSC manufacturing.