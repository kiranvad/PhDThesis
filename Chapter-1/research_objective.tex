The research objective of this thesis is to identify and understand role of representation in the three tasks of accelerated material discovery; namely:~a) down selection;~b) autonomous experimental design;~c) property-response detection. 
Towards this the following research tasks are defined. The cyclic voltammetry responses is used as an example of complex, high-dimensional data with multiple representations possible for a task in hand. Phase diagrams of polymer blends are used as an example to determine design rules for solvent selection with applications in manufacturing of organic solar cells.
\begin{itemize}
    \item {\textbf{Research Task 1 }\textit{Down selection:} This task focuses on building a metric representation that encodes compositional connectivity into the structure of high-dimensional spectral data. We use a combinatorial CV dataset to understand the challenges in the context of obtaining a low-throughput of samples from an expert defined high-throughput search. With a goal to encoding compositional continuity to the high-dimensional structure of the data, we determine a mode diagram that is both useful for down selection and determination of property-response relations.}
    \item{\textbf{Research Task 2 }\textit{Autonomous knowledge extraction:} This task focuses on building stochastic or probabilistic representations of the data in the context of kinetic knowledge extraction from cyclic voltammetry. Extracting rate dependent performance measures of a catalyst is a bottleneck usually governed by unknown, complex underlying mechanisms. Extracting rate dependent performance measures of a (molecular)catalyst is typically a laborious process involving a grid search to identify "ideal" conditions to run an experiment that produces a catalyst response near its kinetic limits. We propose and show an empirical acceleration in data sampling by purely our selection strategy as a search of a particular probabilistic representation of CV curves.}
    \item{\textbf{Research Task 3 }\textit{Property-response map detection: } This task focuses on evaluating polymer mixtures based on their phase diagrams. We apply clustering method as an exploratory data analysis technique to the problem of solvent selection in organic solar cells. This task produces a data-driven alternative for experimental evaluation of compatibility of solvents for a given polymer-blend.}
\end{itemize}

The thesis is an accumulation of three papers~\cite{MLCD}, and it is structured as follows. 
Chapter 2 describes a metric learning approach for down selection of high-throughput spectral data with CV and XRD as examples. 
Chapter 3 describes a probabilistic representation of CV curves for actively searching for target CV shapes that are useful for extracting kinetic information. 
In Chapter 4 we use clustering of phase diagrams as a data driven alternative for solvent based manufacturing of organic solar cells. 
We conclude the thesis in Chapter 5.
