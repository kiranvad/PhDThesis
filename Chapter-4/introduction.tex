\section{Introduction}
Organic thin films made of conjugated polymers and/or small molecules have captured the interest of the organic electronics industry due to their wide range of alterable properties (e.g., color of light emission and solubility in organic solvents). 
Of those properties, solubility crucially facilitates solution-based polymer processing from the liquid phase, making the materials ideal for rapid, inexpensive, large-scale, low-temperature roll-to-roll fabrication. 
Other important properties are flexibility, stretchability, softness, and compatibility with biological systems, all of which are typically absent in conventional silicon-based solutions. 
For these reasons, organic thin films have the potential to revolutionize the next generation of electronic devices within a wide array of technologies, including organic transistors, organic solar cells (OSC), and implantable medical devices and sensors.
At the current state-of-the-art, multiple processing variants exist (e.g., spin coating, doctor blading, casting, roll-to-roll manufacturing), as well as multiple variants tailored for specific materials systems (e.g., blend of polymers and small molecules).  
This proliferation of processing variants is driven by the realization that small changes in the processing can significantly improve a device's performance. 
Classic examples include changing solvents~\cite{Shaheen2001} and adding thermal annealing as a post-processing step, which have together generated organic solar cells with efficiency increased by two orders of magnitude. 
However, processing variants have been all chosen as a result of trial-and-error approaches. 
Consequently, it has remained challenging to establish reliable generalizations of material-process-morphology relationships that can be used to invert those relationships and inform new solvent-based manufacturing variant design.
In this chapter, we use the phase diagram as a representation of the thermodynamics characteristics with the goal of deriving the design rules for solvent selection in organic solar cells. 

Organic Solar Cells~(OSC), the direct application for this work, typically consists of a donor and acceptor materials sandwiched between two electrodes.
Conjugated polymer is typically selected as an electron donor, fullerene (or another polymer) is typically selected as an acceptor. 
OSC are manufactured using various methods of which we are interested in solvent based techniques where donor and acceptor polymers are mixed in a volatile solvent(s)~\cite{krebs2009fabrication}. 
During the manufacturing process, the solvent evaporates to form the morphology. 
Solvent evaporation directs the morphology evolution, and the choice of the solvent is of high importance for the final efficiency of the devices~\cite{Shaheen2001}. 
The first step towards understanding the formation of various phases in the presence of evaporating solvent (of mixture of solvents) is to study the thermodynamics of polymer mixtures. 
In this area, only recently figures of merit have been introduced to capture some characteristics of mixing behavior~\cite{ye2018miscibility}, and to establish a relationship with device properties, followed by derivation of the initial design rules~\cite{ye2018}. Also, features of phase diagrams been included in the reasoning about fabrication of OSC performance~\cite{tashvigh2015novel,li2011determination}, resulting in some initial success. 

In this chapter, we focus on the phase diagram of multi-component material system.
CALPHAD software~\cite{sundman2015opencalphad} is the most widely used tool to study phase diagrams in multi-component systems, but it focuses mostly on alloys. 
Similarly, in the area of fluid multi-component systems, similar approaches have emerged only recently~\cite{SoftMatterCEM,voskov2015ternapi}. 
In the area of organic blends, the phase diagrams of polymeric, or small molecule, multi-component material systems have been analyzed only up to ternaries~\cite{NBB07,li2011determination}. 
This is because the phase diagram construction for multi-component systems has high computational complexity. 
For example, the computational complexity of the equilibrium determination based on convex envelope method (CEM) is
\(\mathcal{O}(N_{\phi}^{(N-1)N/2})\), where \(N\) is the number of components and \(N_{\phi}\) is the number of grid points per each component~\cite{SoftMatterCEM,voskov2015ternapi}.

In this chapter, the CEM method introduced in~\cite{OryllCEM,SoftMatterCEM} is used due to its reasonable computational cost up to
quaternary systems.
We focus on material systems under fixed pressure and temperature, and aim to determine the number of phases for a given a given multi-nary composition.
CEM involves finding the equilibrium compositions of a multi-component system that can be understood by studying the Gibbs free energy landscape~\cite{GibbsCriteria1}. 
A global minimum of the Gibbs free energy landscape determines a true equilibrium state of the system while a meta-stable region is determined by local minima~\cite{OryllCEM,TangentPlaneCriteria,GibbsCriteria1}. 
CEM computes a map of composition space to its corresponding stable phases, given a free energy landscape~(comprising of composition as coordinates and energy as the height value).
The resulting map from CEM when visualized in the composition coordinates-- referred to as the \textit{phase diagram}--reveals the coexisting phases as a function of composition~\cite{SoftMatterCEM}.

We are interested in a high-throughput generation and analysis of material phase diagrams.
In particular, for solvent based OSC manufacturing, a series of experiments needs to be performed in order to select a good solvent for a given set of molecules.
This greatly limits the ability to search and evaluate large amount of solvents in a high-throughput manner.
Alternatively, one can use a phase diagram as a signature of a thermodynamic compatibility of a solvent when mixed with the set of conjugated polymers.
Techniques borrowed from machine learning allows us to study phase diagrams as points in a high-dimensional data space and analyze multiple phase diagrams together using multi-variate approaches.   
In this chapter, multi-variate approaches are used to identify subgroups in phase diagrams, obtain lower dimensional representations of the data space, that are useful in understanding statistical design rules for solvent selection. 

The rest of the chapter is arranged as follows: in \Cref{sec:cem} the convex envelope method~(CEM) to obtain a phase diagram is explained; then the CEM is applied in a high-throughput manner to generate large data sets described in \Cref{sec:htedata}; The data analysis framework is presented in \Cref{sec:pipeline} for the phase diagrams data sets and derivation of data-driven design rules for solvent selection in organic blend manufacturing are demonstrated in \Cref{sec:results}.    
